\documentclass[]{article} 
\usepackage[czech]{babel}
\usepackage[IL2]{fontenc}
\usepackage[utf8]{inputenc}
\usepackage[usenames,dvipsnames]{color}
\usepackage{graphicx}
\usepackage{colortbl}
\usepackage{stmaryrd}
\usepackage{pifont}
\usepackage{hyperref}

\oddsidemargin=-5mm
\evensidemargin=-5mm\marginparwidth=.08in \marginparsep=.01in
\marginparpush=5pt\topmargin=-15mm\headheight=12pt
%\headsep=25pt\footheight=12pt \footskip=30pt\textheight=25cm
\textwidth=17cm\columnsep=2mm
\columnseprule=1pt\parindent=15pt\parskip=2pt

\begin{document}
\begin{center}
\bf Semestralni projekt MI-PAR 2011/2012:\\[5mm]
    Kostra grafu s minimálním stupněm\\[5mm] 
       Luboš Krčál\\
       Alena Varkočková\\[2mm]
magisterské studium, FIT CVUT, Kolejní 550/2, 160 00 Praha 6\\[2mm]
\today
\end{center}

\section{Definice problému a popis sekvenčního algoritmu}

\subsection{Vstupní data}
$G(V,E)$ = jednoduchý neorientovaný neohodnocený k-regulární graf o~$n$ uzlech a $m$ hranách\\
$n$ = přirozené číslo představující počet uzlů grafu $G$, $n \geq 5$\\
$k$ = přirozené číslo řádu jednotek představující stupeň uzlu grafu $G$, $n \geq k \geq 3$; $n$ a $k$~nejsou současně obě liché\\
\\
Doporučení pro generování $G$:\\
\\
Usage: Graf byl generován pomocí poskytnutého \href{''https://edux.fit.cvut.cz/courses/MI-PAR/labs/zadani_semestralnich_praci/generator_grafu'}{'generátoru'} grafu s~volbou typu grafu \texttt{"-t REG"}, který vygeneruje souvislý neorientovaný neohodnocený graf.

\subsection{Výstupní data}
Výstupem programu je kosra grafu a její stupeň v následujícím formátu:

\begin{verbatim}
New spanning tree of degree: 3
[0->6][0->7][0->13][6->4][4->3][3->1][1->2][1->8][2->12][3->5][4->10][6->14][7->9][7->11][14->15][15->16][15->17]
\end{verbatim}

\section{Popis paralelniho algoritmu a jeho implementace v MPI}

\section{Namerene vysledky a vyhodnoceni}

\section{Závěr}

\end{document}